\chapter{ตัวอย่างการเขียน LaTeX}
\label{example}

\section{โครงสร้างของ Template}
ในไฟล์ zip ประกอบด้วยไฟล์ต่าง ๆ ดังต่อไปนี้
\begin{enumerate}[1.]
    \item \filename{thesis.tex} เป็นไฟล์หลักที่ประกอบด้วย ไฟล์ย่อยต่างๆ ตลอดจนข้อมูลของวิทยานิพนธ์ เช่น ชื่อวิทยานิพนธ์ทั้งภาษาไทย และอังกฤษ ชื่อผู้แต่ง ชื่อกรรมการสอบวิทยานิพนธ์
    \item \filename{info.tex} - ไฟล์ที่กำหนดข้อมูลของโครงงาน
    \item \filename{abstract_th.tex} - ไฟล์ที่เขียนบทคัดย่อ ภาษาไทย
    \item \filename{abstract_en.tex} - ไฟล์ที่เขียนบทคัดย่อ ภาษาอังกฤษ
    \item \filename{acronyms.tex} - ไฟล์ที่เขียนสัญลักษณ์ หรือคำย่อ (ถ้าใช้)
    \item \filename{ack_th.tex} - ไฟล์ที่เขียนกิตติกรรมประกาศ ภาษาไทย
    \item \filename{ack_en.tex} - ไฟล์เขียนกิตติกรรมประกาศ ภาษาอังกฤษ (ไม่ใช้ไฟล์นี้)
    \item \filename{chapter[1..5].tex} - ไฟล์ของบทต่าง ๆ ศึกษารายละเอียดของบทต่าง ๆ ในคู่มือวิทยานิพนธ์~\cite{tu:2556fk}
    \item \filename{appendix[A..B].tex} - ไฟล์ภาคผนวก %% โดยภาคผนวกมีได้สูงสุด 10 ภาคผนวก
    \item \filename{refs.bib} - ไฟล์ฐานข้อมูลรายการอ้างอิง (Bibliography) ในรูปแบบ BibTeX
    \item \filename{TUthesis.sty} - ไฟล์ template style สำหรับมหาวิทยาลัยธรรมศาสตร์ (ไม่จำเป็นต้องแก้ไขไฟล์นี้)
    \item \filename{mystyle.sty} - ไฟล์ template style สำหรับปรับแต่งรายงานโครงงาน (ไม่จำเป็นต้องแก้ไขไฟล์นี้)
    \item \filename{example.tex} - ไฟล์ตัวอย่างการเขียน LaTeX
\end{enumerate}

\newpage
\section{การเขียน LaTeX}
สามารถอ่านจากบทความต่าง ๆ
\begin{enumerate}[1.]
    \item บทแนะนำ LaTeX 2e แบบไม่ค่อยย่อ แปลโดย คุณจักรภาษณ์ วิศวกุล\\
    \url{http://zelmanov.ptep-online.com/ctan/lshort_thai.pdf}

    %%\item ตัวอย่างคำสั่ง LaTeX ของภาควิชาคณิตศาสตร์ คณะวิทยาศาสตร์ ม. เชียงใหม่\\
    %% Link down
    %%\url{http://www.math.science.cmu.ac.th/thanasak/pow/CMU_LaTeX_2010_All.pdf} %%
    \item บทแนะนำ LaTeX 2e แบบไม่ค่อยย่อ ภาษาอังกฤษ โดย Tobias Oetiker\\
    \url{https://tobi.oetiker.ch/lshort/lshort.pdf}

    \item ตัวอย่างคำสั่ง LaTeX ของภาควิชาคณิตศาสตร์ คณะวิทยาศาสตร์ ม. เกษตรศาสตร์\\
    \url{http://maths.sci.ku.ac.th/th/service/training/2556/2555-06-17_1/doc_xelatex.pdf}

    \item Thesis template และแนะนำการใช้ LaTeX ของอาจารย์ \mbox{ดร. ฑิตยา หวานวารี}  ภาควิชาคณิตศาสตร์และวิทยาการคอมพิวเตอร์ คณะวิทยาศาสตร์ จุฬาลงกรณ์มหาวิทยาลัย\\
    \url{http://pioneer.netserv.chula.ac.th/~wdittaya/}

    \item LaTeX ใน Wikibook\\
    \url{https://en.wikibooks.org/wiki/LaTeX}

    \item การใช้ Endnote กับ Latex และ BibTeX\\
    \url{http://www.rhizobia.co.nz/latex/convert}
\end{enumerate}

\newpage
\section{การใช้ฟอนต์}
\
ฟอนต์หลักที่ได้กำหนดไว้แล้วในรายงานคือ \texttt{TH Sarabun New} การใช้งานฟอนต์ สามารถเลือกรูปแบบการแสดงได้ดังนี้

\begin{table}[H]
    \centering
    \footnotesize
    \begin{threeparttable}
        \caption{รูปแบบการแสดงฟอนต์}
        \label{tab:font style}
        \begin{tabular}{l c C{6cm} c}
            \toprule
            \textbf{รูปแบบ} & \textbf{คำสั่ง} & \textbf{ตัวอย่างการใช้คำสั่ง} & \textbf{ผลการแสดง} \\
            \midrule
            ตัวหนา & \verb|\textbf{text}| & \verb|\textbf{|พลังของแผ่นดิน\verb|}| & \textbf{พลังของแผ่นดิน} \\
            ตัวเอียง & \verb|\textit{text}| & \verb|\textit{|พลังของแผ่นดิน\verb|}| & \textit{พลังของแผ่นดิน} \\
            ตัวเอียงและหนา & \verb|\textit{\textbf{text}}| & \verb|\textit{\textbf{|พลังของแผ่นดิน\verb|}}| & \textit{\textbf{พลังของแผ่นดิน}} \\
            ตัวหนาและเอียง & \verb|\textbf{\textit{text}}| & \verb|\textbf{\textit{|พลังของแผ่นดิน\verb|}}| & \textbf{\textit{พลังของแผ่นดิน}} \\
            ขีดเส้นใต้ & \verb|\underline{text}| & \verb|\underline{|พลังของแผ่นดิน\verb|}| & \underline{พลังของแผ่นดิน} \\
            ตัวหนาและเอียง\footnotetext[1]{test} & \verb|\underline{\textit{text}}| & \verb|\underline{\textit{|พลังของแผ่นดิน\verb|}}| & \underline{\textit{พลังของแผ่นดิน}} \\
            เน้นข้อความ\tnote{1} & \verb|\emph{text}| & \verb|\emph{|พลังของแผ่นดิน\verb|}| & \emph{พลังของแผ่นดิน} \\
            \bottomrule
        \end{tabular}
        \begin{tablenotes}
            \item [1] ปกติจะแสดงเป็นแบบเดียวกับตัวเอียง (italic)
        \end{tablenotes}
    \end{threeparttable}
\end{table}

LaTeX มีคำสั่งในการเปลี่ยนขนาดฟอนต์ แต่สำหรับในรายงานโครงงาน ไม่จำเป็นต้องเปลี่ยนขนาดฟอนต์ คำสั่งใน \autoref{tab:font size} เหล่านี้จะเปลี่ยนขนาดของฟอนต์ที่ใช้ในสภาพแวดล้อม\mbox{ปัจจุบัน} จนกว่าจะมีการใช้คำสั่งเปลี่ยนขนาดใหม่

\begin{table}[H]
    \centering
    \footnotesize
    \begin{threeparttable}
        \caption{เปลี่ยนขนาดของฟอนต์}
        \label{tab:font size}
        \begin{tabular}{l C{6cm} c}
            \toprule
            \textbf{คำสั่ง} & \textbf{ตัวอย่างการใช้คำสั่ง} & \textbf{ผลการแสดง} \\
            \midrule
            \verb|\tiny| & \verb|\tiny| พลังของแผ่นดิน & \tiny พลังของแผ่นดิน \\
            \verb|\scriptsize| & \verb|\scriptsize|พลังของแผ่นดิน| & \scriptsize พลังของแผ่นดิน \\
            \verb|\footnotesize| & \verb|\footnotesize| พลังของแผ่นดิน & \footnotesize พลังของแผ่นดิน \\
            \verb|\small| & \verb|\small| พลังของแผ่นดิน & \small พลังของแผ่นดิน \\
            \verb|\normalsize| & \verb|\normalsize|พลังของแผ่นดิน & \normalsize พลังของแผ่นดิน \\
            \verb|\large| & \verb|\large| พลังของแผ่นดิน & \large พลังของแผ่นดิน \\
            \verb|\Large| & \verb|\Large| พลังของแผ่นดิน & \Large พลังของแผ่นดิน \\
            \verb|\LARGE| & \verb|\LARGE| พลังของแผ่นดิน & \LARGE พลังของแผ่นดิน \\
            \verb|\huge| & \verb|\huge| พลังของแผ่นดิน & \huge พลังของแผ่นดิน \\
            \verb|\Huge| & \verb|\Huge| พลังของแผ่นดิน & \Huge พลังของแผ่นดิน \\
            \bottomrule
        \end{tabular}
    \end{threeparttable}
\end{table}

ในกรณีที่ต้องการเปลี่ยนขนาดฟอนต์เฉพาะบางส่วนของข้อความ ให้เขียนคำสั่งภายในสภาพแวดล้อมปิด (อยู่ภายในเล็บวงเล็บปีกกา) เช่น \verb|{\footnotesize test}|

\newpage
\section{การเพิ่มรายการ (List)}

รูปแบบคำสั่งในการเขียนรายการใน LaTeX คือ

\begin{lstlisting}[caption={รูปแบบคำสั่งในการเขียนรายการ (list)}, label={list:list format},language=TeX]
\begin{|\textcolor{blue}{list\_type}|}
    \item รายการที่ 1
    \item รายการที่ 2
    \item รายการที่ 3
\end{|\textcolor{blue}{list\_type}|}
\end{lstlisting}

โดย \textcolor{blue}{\texttt{list\_type}} ใน \autoref{list:list format} มีอยู่ 3 แบบคือ

\begin{itemize}[noitemsep]
    \item \textbf{\color{blue}itemize} รายการแบบใช้สัญลักษณ์ (เช่น bullet point)
    \item \textbf{\color{blue}enumerate} รายการแบบใช้ตัวเลข (หรือตัวอักษร เช่น 1. (a))
    \item \textbf{\color{blue}description} รายการแบบการอธิบายคำ
\end{itemize}

%%=================================================
%% Itemize
%%=================================================

\subsection{ตัวอย่างรายการแบบ \textcolor{blue}{itemize}}

\begin{exampleBox}[sidebyside]{รายการแบบ defaut (bullet)}
\begin{lstlisting}[frame=none,language={[LaTeX]TeX}]
\begin{itemize}
    \item รายการที่ 1
    \item รายการที่ 2
    \item รายการที่ 3
\end{itemize}
\end{lstlisting}
\tcblower
\begin{itemize}[noitemsep]
    \item รายการที่ 1
    \item รายการที่ 2
    \item รายการที่ 3
\end{itemize}
\end{exampleBox}


\begin{exampleBox}[sidebyside]{รายการซ้อนกัน (nested list)}
\small
\begin{lstlisting}[frame=none,language={[LaTeX]TeX}]
\begin{itemize}
    \item รายการระดับ 1
    \begin{itemize}
        \item รายการระดับ 2
        \begin{itemize}
            \item รายการระดับ 3
            \begin{itemize}
                \item รายการระดับ 4
            \end{itemize}
        \end{itemize}
    \end{itemize}
\end{itemize}
\end{lstlisting}
\tcblower
\small
\begin{itemize}[noitemsep,leftmargin=*,label=\textbullet]
    \item รายการระดับ 1
    \begin{itemize}
        \item รายการระดับ 2
        \begin{itemize}
            \item รายการระดับ 3
            \begin{itemize}
                \item รายการระดับ 4
            \end{itemize}
        \end{itemize}
    \end{itemize}
\end{itemize}
\end{exampleBox}


\begin{exampleBox}[lefthand ratio=0.55,sidebyside]{เปลี่ยนสัญลักษณ์ที่ระดับต่าง ๆ}
\small
\begin{lstlisting}[frame=none,language={[LaTeX]TeX}]
\begin{itemize}[\textbullet]
    \item รายการระดับ 1
    \begin{itemize}[\textendash]
        \item รายการระดับ 2
        \begin{itemize}[$\ast$]
            \item รายการระดับ 3
            \begin{itemize}[$\square$]
                \item รายการระดับ 4
            \end{itemize}
        \end{itemize}
    \end{itemize}
\end{itemize}
\end{lstlisting}
\tcblower
\small
\begin{itemize}[\textbullet,noitemsep,leftmargin=*,]
    \item รายการระดับ 1
    \begin{itemize}[\textendash]
        \item รายการระดับ 2
        \begin{itemize}[$\ast$]
            \item รายการระดับ 3
            \begin{itemize}[$\square$]
                \item รายการระดับ 4
            \end{itemize}
        \end{itemize}
    \end{itemize}
\end{itemize}
\end{exampleBox}

%%=================================================
%% Enumerate
%%=================================================

\subsection{ตัวอย่างรายการแบบ \textcolor{blue}{enumerate}}

\begin{exampleBox}[sidebyside]{รายการมีลำดับแบบ defaut (ลำดับเป็นตัวเลขอารบิก)}
\begin{lstlisting}[frame=none,language={[LaTeX]TeX}]
\begin{enumerate}
    \item รายการที่ 1
    \item รายการที่ 2
    \item รายการที่ 3
\end{enumerate}
\end{lstlisting}
\tcblower
\begin{enumerate}[noitemsep]
    \item รายการที่ 1
    \item รายการที่ 2
    \item รายการที่ 3
\end{enumerate}
\end{exampleBox}

\begin{exampleBox}[sidebyside]{รายการซ้อนกัน (nested list)}
\small
\begin{lstlisting}[frame=none,language={[LaTeX]TeX}]
\begin{enumerate}
    \item รายการระดับ 1
    \begin{enumerate}
        \item รายการระดับ 2
        \begin{enumerate}
            \item รายการระดับ 3
            \begin{enumerate}
                \item รายการระดับ 4
            \end{enumerate}
        \end{enumerate}
    \end{enumerate}
\end{enumerate}
\end{lstlisting}
\tcblower
\small
\begin{enumerate}
    \item รายการระดับ 1
    \begin{enumerate}
        \item รายการระดับ 2
        \begin{enumerate}
            \item รายการระดับ 3
            \begin{enumerate}
                \item รายการระดับ 4
            \end{enumerate}
        \end{enumerate}
    \end{enumerate}
\end{enumerate}
\end{exampleBox}

\begin{exampleBox}[sidebyside]{เปลี่ยนหมายเลขลำดับที่ระดับต่าง ๆ}
\small
\begin{lstlisting}[frame=none,language={[LaTeX]TeX}]
\begin{enumerate}[1.]
    \item รายการระดับ 1
    \item รายการระดับ 1
    \item รายการระดับ 1
    \begin{enumerate}[(a)]
        \item รายการระดับ 2
        \item รายการระดับ 2
        \item รายการระดับ 2
        \begin{enumerate}[A)]
            \item รายการระดับ 3
            \item รายการระดับ 3
            \item รายการระดับ 3
            \begin{enumerate}[i.]
                \item รายการระดับ 4
                \item รายการระดับ 4
                \item รายการระดับ 4
            \end{enumerate}
        \end{enumerate}
    \end{enumerate}
\end{enumerate}
\end{lstlisting}
\tcblower
\small
\begin{enumerate}[1.]
    \item รายการระดับ 1
    \item รายการระดับ 1
    \item รายการระดับ 1
    \begin{enumerate}[(a)]
        \item รายการระดับ 2
        \item รายการระดับ 2
        \item รายการระดับ 2
        \begin{enumerate}[A)]
            \item รายการระดับ 3
            \item รายการระดับ 3
            \item รายการระดับ 3
            \begin{enumerate}[i.]
                \item รายการระดับ 4
                \item รายการระดับ 4
                \item รายการระดับ 4
            \end{enumerate}
        \end{enumerate}
    \end{enumerate}
\end{enumerate}
\end{exampleBox}


\begin{exampleBox}[lefthand ratio=0.55,sidebyside]{เปลี่ยนหมายเลขลำดับที่ระดับต่าง ๆ โดยใช้หมายเลขหัวข้อ}
\scriptsize
\begin{lstlisting}[frame=none,language={[LaTeX]TeX}]
\begin{enumerate}[\thesection.\arabic*]
    \item รายการระดับ 1
    \item รายการระดับ 1
    \item รายการระดับ 1
    \begin{enumerate}[\theenumi.\arabic*]
        \item รายการระดับ 2
        \item รายการระดับ 2
        \item รายการระดับ 2
        \begin{enumerate}[\theenumii.\arabic*]
            \item รายการระดับ 3
            \item รายการระดับ 3
            \item รายการระดับ 3
            \begin{enumerate}[\theenumiii.\arabic*]
                \item รายการระดับ 4
                \item รายการระดับ 4
                \item รายการระดับ 4
            \end{enumerate}
        \end{enumerate}
    \end{enumerate}
\end{enumerate}
\end{lstlisting}
\tcblower
\scriptsize
\begin{enumerate}[\thesection.\arabic*]
    \item รายการระดับ 1
    \item รายการระดับ 1
    \item รายการระดับ 1
    \begin{enumerate}[\theenumi.\arabic*]
        \item รายการระดับ 2
        \item รายการระดับ 2
        \item รายการระดับ 2
        \begin{enumerate}[\theenumii.\arabic*]
            \item รายการระดับ 3
            \item รายการระดับ 3
            \item รายการระดับ 3
            \begin{enumerate}[\theenumiii.\arabic*]
                \item รายการระดับ 4
                \item รายการระดับ 4
                \item รายการระดับ 4
            \end{enumerate}
        \end{enumerate}
    \end{enumerate}
\end{enumerate}
\end{exampleBox}
\newpage
\section{การเพิ่มรูป}

\begin{exampleBox}[lefthand ratio=0.7,sidebyside]{เพิ่มรูปโดยใช้สภาพแวดล้อม \texttt{figure}}
\footnotesize
\begin{lstlisting}[frame=none]
\begin{figure}[H]
    \centering
    \includegraphics[width=2cm]{figures/tu-logo-bw.jpg}
    \caption{ตราธรรมจักร}
    \label{fig:thammajak}
\end{figure}
\end{lstlisting}
\tcblower
\footnotesize
\begin{figure}[H]
    \centering
    \includegraphics[width=2cm]{figures/tu-logo-bw.jpg}
    \caption{ตราธรรมจักร}
    \label{fig:thammajak}
\end{figure}
\end{exampleBox}

การเขียนคำอธิบายรูป ในกรณีที่เป็นรูปที่สร้างขึ้นเอง ไม่จำเป็นต้องระบุแหล่งที่มา ให้ใช้คำสั่ง \latex{\caption{figure description}} เช่น ใน \autoref{fig:thammajak} ให้ใช้คำสั่ง \verb|\caption{|ตราธรรมจักร\verb|}|

วิธีการอ้างอิงถึงรูป ให้ใช้คำสั่ง \latex{\ref{label}} หรือ \latex{\autoref{label}} เช่น หากต้องการอ้างอิงถึง \autoref{fig:thammajak} ข้างต้น ให้พิมพ์ รูปที่ \latex{\ref{fig:thammajak}} หรือ \latex{\autoref{fig:thammajak}}


\begin{figure}[H]
    \centering
    \includegraphics[width=2cm]{figures/tu-logo-bw.jpg}
    \captionsource{ตราธรรมจักร}{\url{https://tu.ac.th/thammasat-identity}}
    \label{fig:thammajak 2}
\end{figure}

แต่ถ้าได้รูปมาจากที่อื่น ให้ระบุแหล่งที่มาด้วย โดยให้ใช้คำสั่ง \latex{\captionsource{figure description}{figure source}} เช่น หากต้องการระบุแหล่งที่มาใน \autoref{fig:thammajak 2} ให้พิมพ์คำสั่ง \\ \verb|\captionsource{|ตราธรรมจักร\verb|}{\url{https://tu.ac.th/thammasat-identity}}|

\begin{exampleBox}{แสดงหลายรูปย่อยโดยใช้สภาพแวดล้อม \texttt{subfigure}}
\footnotesize
\begin{lstlisting}[frame=none]
\begin{figure}[H]
    \centering
    \begin{subfigure}[b]{0.45\textwidth}
        \centering
        \includegraphics[width=1.5in]{tu-logo-bw}
        \caption{|ตราธรรมจักร ขาวดำ|}
        \label{fig:thammajak bw}
    \end{subfigure}
    ~ %add desired spacing between images, e. g. ~, \quad, \qquad, \hfill etc.
      %(or a blank line to force the subfigure onto a new line)
    \begin{subfigure}[b]{0.45\textwidth}
        \centering
        \includegraphics[width=1.5in]{tu-logo-color}
        \caption{|ตราธรรมจักร สี|}
        \label{fig:thammajak color}
    \end{subfigure}
    \caption{|ตราธรรมจักร ขาวดำและสี|}\label{fig:thammajak both color}
\end{figure}
\end{lstlisting}
\tcblower
\footnotesize
\begin{figure}[H]
    \centering
    \begin{subfigure}[b]{0.45\textwidth}
        \centering
        \includegraphics[width=1.5in]{tu-logo-bw}
        \caption{ตราธรรมจักร ขาวดำ}
        \label{fig:thammajak bw}
    \end{subfigure}
    ~ %add desired spacing between images, e. g. ~, \quad, \qquad, \hfill etc.
    %(or a blank line to force the subfigure onto a new line)
    \begin{subfigure}[b]{0.45\textwidth}
        \centering
        \includegraphics[width=1.5in]{tu-logo-color}
        \caption{ตราธรรมจักร สี}
        \label{fig:thammajak color}
    \end{subfigure}
    \caption{ตราธรรมจักร ขาวดำและสี}\label{fig:thammajak both color}
\end{figure}
\end{exampleBox}


\newpage
\section{การเพิ่มตาราง}

\begin{exampleBox}{เพิ่มตาราง ใช้เส้นขอบขั้น}
\begin{lstlisting}[frame=none,escapeinside={!}{!}]
\begin{table}[H]
    \centering
    \caption{!ตัวอย่างตาราง!}
    \label{tab:table 1}
    \begin{tabular}{| c | c | p{3cm} | p{3cm} |}
        \hline
        !หัวตาราง 1! & !หัวตาราง 1! & !หัวตาราง 1! & !หัวตาราง 1! \\
        \hline
        ข้อมูล & ข้อมูล & ข้อมูล & ข้อมูล \\
        \hline
        ข้อมูล & ข้อมูล & ข้อมูล & ข้อมูล \\
        \hline
        ข้อมูล & ข้อมูล & ข้อมูล & ข้อมูล \\
        \hline
    \end{tabular}
\end{table}
\end{lstlisting}
\tcblower
\begin{table}[H]
    \centering
    \caption{ตัวอย่างตาราง}
    \label{tab:table 1}
    \begin{tabular}{| c | c | p{3cm} | p{3cm} |}
        \hline
        หัวตาราง 1 & หัวตาราง 1 & หัวตาราง 1 & หัวตาราง 1 \\
        \hline
        ข้อมูล & ข้อมูล & ข้อมูล & ข้อมูล \\
        \hline
        ข้อมูล & ข้อมูล & ข้อมูล & ข้อมูล \\
        \hline
        ข้อมูล & ข้อมูล & ข้อมูล & ข้อมูล \\
        \hline
    \end{tabular}
\end{table}
\end{exampleBox}

\newpage
\begin{exampleBox}{เพิ่มตาราง}
\begin{lstlisting}[frame=none]
\begin{table}
    \centering
    \caption{|ตัวอย่างตาราง|}
    \label{tab:table 2}
    \begin{tabular}{c c p{3cm} p{3cm}}
        \toprule
        |หัวตาราง 1| & |หัวตาราง 1| & |หัวตาราง 1| & |หัวตาราง 1| \\
        \midrule
        ข้อมูล & ข้อมูล & ข้อมูล & ข้อมูล \\
        ข้อมูล & ข้อมูล & ข้อมูล & ข้อมูล \\
        ข้อมูล & ข้อมูล & ข้อมูล & ข้อมูล \\
        \bottomrule
    \end{tabular}
\end{table}
\end{lstlisting}
\tcblower
\begin{table}[H]
    \centering
    \caption{ตัวอย่างตาราง ไม่มีเส้นขอบขั้น}
    \label{tab:table 2}
    \begin{tabular}{c c p{3cm} p{3cm}}
        \toprule
        หัวตาราง 1 & หัวตาราง 1 & หัวตาราง 1 & หัวตาราง 1 \\
        \midrule
        ข้อมูล & ข้อมูล & ข้อมูล & ข้อมูล \\
        ข้อมูล & ข้อมูล & ข้อมูล & ข้อมูล \\
        ข้อมูล & ข้อมูล & ข้อมูล & ข้อมูล \\
        \bottomrule
    \end{tabular}
\end{table}
\end{exampleBox}

\newpage
\section{การอ้างอิง (รายการอ้างอิง)}
รูปแบบที่ตั้งไว้ใน template นี้ โดยคำสั่งที่กำหนดไว้ ใน \filename{mystyle.sty} คือ
\begin{enumerate}
    \item \latex{\usepackage[backend=biber,style=ieee,bibencoding=utf8]{biblatex}}
    \item \latex{\addbibresource{refs.bib}} โดย \filename{refs.bib} คือไฟล์ฐานข้อมูลรายการอ้างอิงในรูปแบบ BibTeX
    \item เวลาอ้างอิง ให้ใช้คำสั่ง \latex{\cite{key}}
\end{enumerate}

\subsection{ตัวอย่างการอ้างอิง}

\subsubsection{การอ้างอิงหนังสือ}

ให้เขียนข้อมูลรายการของหนังสือที่ต้องการอ้างอิง เก็บไว้ในไฟล์ \filename{refs.bib} ตัวอย่างเช่น \autoref{list:ref book}

\begin{lstlisting}[caption={ตัวอย่างการอ้างอิงหนังสือ},label={list:ref book}]
@book{Oversampling:1992,
    editor        = "J. C. Candy and G. C. Temes",
    title         = "Oversampling Delta-Sigma Data Converters Theory,
                     Design and Simulation",
    publisher     = "{IEEE} Press.",
    location      = "New York",
    year          = "1992"
}
\end{lstlisting}

ชื่อที่ต้องใช้ในการอ้างอิงสำหรับ \autoref{list:ref book} คือ \latex{Oversampling:1992} โดยเราสามารถ\mbox{กำหนด}ชื่อนี้ได้เอง การอ้างอิงใน \autoref{list:ref book} ให้ใช้คำสั่ง \latex{\cite{Oversampling:1992}} \cite{Oversampling:1992}


\begin{lstlisting}[caption={ตัวอย่างการอ้างอิงบทความ},label={list:ref article}]
@article{Midgap:1997,
    author        = "A. Castaldini and A. Cavallini and B. Fraboni
                     and P. Fernandez and J. Piqueras",
    title         = "Midgap Traps Related to Compensation Processes in
                     {CdTe} Alloys",
    journaltitle  = "Phys. Rev. B.",
    volume        = "56",
    number        = "23",
    year          = "1997",
    pages         = "14897-14900"
}
\end{lstlisting}

สำหรับข้อมูลใน \autoref{list:ref article} ชื่อที่จะใช้ในการอ้างอิงคือ \latex{Midgap:1997} โดยการ\mbox{อ้างอิง} ให้ใช้คำสั่ง \latex{\cite{Midgap:1997}} \cite{Midgap:1997}

\subsubsection{การอ้างอิงเอกสารออนไลน์}

\begin{lstlisting}[caption={ตัวอย่างการอ้างอิงเอกสารออนไลน์ 1},label={list:ref online 1}]
@misc{NFC,
author        = "|{เสฏฐวุฒิ แสนนาม}|",
title         = "{NFC}",
year          = "2013",
howpublished  = "|\url{https://www.thaicert.or.th/papers/general/2013/pa2013ge001.html}|",
}
\end{lstlisting}

การอ้างอิงใน \autoref{list:ref online 1} ให้ใช้คำสั่ง \latex{\cite{NFC}} \cite{NFC}


